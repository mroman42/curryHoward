%%%
% Apuntes para un seminario sobre el isomorfismo Curry-Howard
% Author: Mario Román
%
% License:
%  CC BY-NC-SA 3.0 (http://creativecommons.org/licenses/by-nc-sa/3.0/)
%%%

%%%
% Este archivo parte de una modificación de una plantilla de Latex para adaptarla al castellano.
% Su autor original es Frits Wenneker (http://www.howtotex.com).
%%%

%%%%%%%%%%%%%%%%%%%%%%%%%%%%%%%%%%%%%%%%
% Short Sectioned Assignment
% LaTeX Template
% Version 1.0 (5/5/12)
%
% This template has been downloaded from:
% http://www.LaTeXTemplates.com
%
% Original author:
% Frits Wenneker (http://www.howtotex.com)
%
% License:
% CC BY-NC-SA 3.0 (http://creativecommons.org/licenses/by-nc-sa/3.0/)
%
%%%%%%%%%%%%%%%%%%%%%%%%%%%%%%%%%%%%%%%%%

%----------------------------------------------------------------------------------------
%	PACKAGES AND OTHER DOCUMENT CONFIGURATIONS
%----------------------------------------------------------------------------------------

\documentclass[paper=a4, fontsize=11pt, spanish]{scrartcl} % A4 paper and 11pt font size
\usepackage[utf8]{inputenc}
\usepackage[spanish]{babel}
\selectlanguage{spanish}

%\usepackage{listings}
\usepackage{listingsutf8}

\usepackage[usenames,dvipsnames]{xcolor}
\definecolor{dkgreen}{rgb}{0,0.35,0}
\definecolor{dkviolet}{rgb}{0.3,0,0.5}
\definecolor{dkred}{rgb}{0.5,0,0}
\definecolor{dkblue}{rgb}{0, 0, 0.9}
\definecolor{ltblue}{rgb}{0.38, 0.55, 0.8}
\usepackage{lstcoq}

\usepackage{catchfilebetweentags}
\usepackage{verbatim}
\usepackage{bussproofs}

\usepackage[hidelinks]{hyperref}

\newenvironment{coq_example}{\begin{coq}}{\end{coq}}

% Tildes
\lstset{
     literate=%
         {á}{{\'a}}1
         {í}{{\'i}}1
         {é}{{\'e}}1
         {ý}{{\'y}}1
         {ú}{{\'u}}1
         {ó}{{\'o}}1
         {ě}{{\v{e}}}1
         {š}{{\v{s}}}1
         {č}{{\v{c}}}1
         {ř}{{\v{r}}}1
         {ž}{{\v{z}}}1
         {ď}{{\v{d}}}1
         {ť}{{\v{t}}}1
         {ň}{{\v{n}}}1                
         {ů}{{\r{u}}}1
         {Á}{{\'A}}1
         {Í}{{\'I}}1
         {É}{{\'E}}1
         {Ý}{{\'Y}}1
         {Ú}{{\'U}}1
         {Ó}{{\'O}}1
         {Ě}{{\v{E}}}1
         {Š}{{\v{S}}}1
         {Č}{{\v{C}}}1
         {Ř}{{\v{R}}}1
         {Ž}{{\v{Z}}}1
         {Ď}{{\v{D}}}1
         {Ť}{{\v{T}}}1
         {Ň}{{\v{N}}}1                
         {Ů}{{\r{U}}}1    
}


\usepackage[T1]{fontenc} % Use 8-bit encoding that has 256 glyphs
\usepackage{fourier} % Use the Adobe Utopia font for the document - comment this line to return to the LaTeX default



\usepackage{amsmath,amsfonts,amsthm} % Math packages
\usepackage{tabularx}

\usepackage{sectsty} % Allows customizing section commands
\allsectionsfont{\centering \normalfont\scshape} % Make all sections centered, the default font and small caps

\usepackage{fancyhdr} % Custom headers and footers
\pagestyle{fancyplain} % Makes all pages in the document conform to the custom headers and footers
\fancyhead{} % No page header - if you want one, create it in the same way as the footers below
\fancyfoot[L]{} % Empty left footer
\fancyfoot[C]{} % Empty center footer
\fancyfoot[R]{\thepage} % Page numbering for right footer
\renewcommand{\headrulewidth}{0pt} % Remove header underlines
\renewcommand{\footrulewidth}{0pt} % Remove footer underlines
\setlength{\headheight}{13.6pt} % Customize the height of the header

\numberwithin{equation}{section} % Number equations within sections (i.e. 1.1, 1.2, 2.1, 2.2 instead of 1, 2, 3, 4)
\numberwithin{figure}{section} % Number figures within sections (i.e. 1.1, 1.2, 2.1, 2.2 instead of 1, 2, 3, 4)
\numberwithin{table}{section} % Number tables within sections (i.e. 1.1, 1.2, 2.1, 2.2 instead of 1, 2, 3, 4)

\setlength\parindent{0pt} % Removes all indentation from paragraphs - comment this line for an assignment with lots of text

%----------------------------------------------------------------------------------------
%	TITLE SECTION
%----------------------------------------------------------------------------------------

\newcommand{\horrule}[1]{\rule{\linewidth}{#1}} % Create horizontal rule command with 1 argument of height

\title{
\normalfont \normalsize
\textsc{Universidad de Granada} \\ [25pt] % Your university, school and/or department name(s)
\horrule{0.5pt} \\[0.4cm] % Thin top horizontal rule
\huge Curry-Howard \\ % The assignment title
\horrule{2pt} \\[0.5cm] % Thick bottom horizontal rule
}

\author{Mario Román} % Your name

\date{\normalsize\today} % Today's date or a custom date



\definecolor{lgray}{rgb}{0.6,0.6,0.6}
\definecolor{gray}{rgb}{0.5,0.5,0.5}
\definecolor{mauve}{rgb}{0.58,0,0.82}



\begin{document}

%----------------------------------------------------------------------------------------
% Escribir una ecuación:
%  \begin{align}
%  \begin{split}
%----------------------------------------------------------------------------------------

\lstset{frame=tb,
  language=Haskell,
  aboveskip=3mm,
  belowskip=3mm,
  showstringspaces=false,
  columns=flexible,
  basicstyle={\small\ttfamily},
  numbers=none,
  numberstyle=\tiny\color{gray},
  keywordstyle=\color{blue},
  commentstyle=\color{lgray},
  stringstyle=\color{mauve},
  breaklines=true,
  breakatwhitespace=true
  tabsize=3
}

\maketitle
  El objetivo de estos apuntes es poner de manifiesto el isomorfismo de Curry-Howard y presentar con él
  el asesor de demostraciones Coq. Al lector sólo se le requiere un conocimiento
  superficial de programación funcional.

  \section*{Preludio: Conjuntos y proposiciones}
    Las proposiciones son conjuntos, y los conjuntos son proposiciones. Este isomorfismo
    es trivial para el que estudia matemáticas, pero es potente porque expresa que esas
    dos estructuras son esencialmente, la misma cosa.
    \begin{align*}
     x \in A && P x \\
     x \in A^c && \neg P x \\
     x \in A \cup B && (P x) \vee (Q x) \\
     x \in A \cap B && (P x) \wedge (Q x) \\
     A \subset B && P \Rightarrow Q
    \end{align*}
    De hecho puede construirse el isomorfismo como:
    \begin{alignat*}{4}
    f :\;& \mathtt{Prop} \;& \rightarrow \;& \mathtt{Set}        \\
	\;& P x           \;& \mapsto     \;&  \left\{\; x \;|\; P x\; \right\}      
    \end{alignat*}
    \begin{alignat*}{4}
    f^{-1} :\;& \mathtt{Set} \;&  \rightarrow  \;& \mathtt{Prop} \\
	    \;& A            \;&   \mapsto    \;& (Px \Leftrightarrow x \in A)
    \end{alignat*}

    Lo realmente interesante del isomorfismo es que preserva las propiedades profundas entre
    los dos objetos. Planteamos la paradoja de Russel en uno de los dos lados y obtenemos
    la paradoja de Quine en el otro.
    \begin{table}[htbp]
      \centering
      \begin{tabularx}{\textwidth}{X|X}
      \textbf{Paradoja de Russell} & \textbf{Paradoja de Quine} \\
      \hline
      El conjunto de los conjuntos que no se contienen a sí 
      mismos no se contiene a sí mismo. &
      Es falso precedido de sí mismo es falso precedido de sí mismo. \\
	\centerline{$m = \{x | x \notin x\}$} &
	\centerline{$M x \equiv \neg x x$} \\  
	\centerline{$m \in m \quad(?)$} &
	\centerline{$M M \quad(?)$} \\ 
      \end{tabularx}
    \end{table}
    
    
  \section{El isomorfismo Curry-Howard}
    %% Una presentación correcta de estos apuntes se cuidará de no enunciar el teorema 
    %% hasta que no sea completamente obvio para la audiencia.
    \begin{center}
      \textit{ Los tipos son teoremas, los programas son demostraciones.}
    \end{center}
    
    En Haskell es fácil encontrar, para tipos \textit{arbitrarios}, funciones que tengan la forma
    general:
    \begin{lstlisting}
  a -> (b -> a)
  (a,b) -> a
  a -> (a -> b) -> b
    \end{lstlisting}
    Mientras que es imposible  encontrar funciones de tipos como:
    \begin{lstlisting}
  a -> (a -> b)
  a -> b
  b -> (b,c)
    \end{lstlisting}
    En este contexto es natural preguntarse qué tipos están poblados (existen elementos de
    ese tipo) y cuales no. \footnote{Realmente no es exactamente imposible poblar un tipo cualquiera: Haskell permite salirse un poco
    del isomorfismo usando funciones recursivas que no terminan o usando la constante \texttt{undefined}.
    Pero no consideraremos esos casos límites.} La respuesta a esta pregunta la da el isomorfismo de Curry-Howard:
    \begin{center}
     \textit{Existe un objeto del tipo T si y sólo si T, interpretado como proposición lógica,
      es cierto.}
    \end{center}
    
    
    %% Hay un problemilla viendo Curry-Howard en Haskell.
    %% Por un lado, podemos tener funciones recursivas que hacen cosas raras: fix f = f fix f
    %% Y por otro lado, podemos tener el problema de que no hay forma elegante de definir False.
    \subsection{Implicatio}
      Comprobamos que si tomamos el operador \texttt{->} como la implicación lógica, obtenemos
      resultados coherentes:
      \begin{lstlisting}
  -- Tautología
  idem :: a -> a
  idem x = x

  -- Podemos aplicar el modus ponens
  modusPonens :: a -> (a -> b) -> b 
  modusPonens x f = f x

  -- Pero no podemos encontrar funciones de este tipo
  nop :: a -> (a -> b)
  nop x = ??
      \end{lstlisting}
      La belleza del isomorfismo queda patente al probar que el modus ponens no es ni más ni 
      menos que la aplicación de funciones.
      
      
    %% ¡El producto categórico era precisamente este en Hask y Prop!  
    \subsection{Conjunctio}
      Y comprobemos ahora que la conjunción lógica se corresponde con el par de tipos \texttt{(\_,\_)}.
      \begin{lstlisting}
  -- Introducción de conjunción
  conjIntro :: a -> b -> (a,b) 
  conjIntro x y = (x,y)

  -- Eliminación de conjunción
  fst :: (a,b) -> a
  snd :: (a,b) -> b
      \end{lstlisting}
      De hecho, el tipo \texttt{(a,b)} está poblado si y sólo si lo están \texttt{a} y \texttt{b}.
      
    \subsection{Disiunctio}
      Necesitamos un tipo que esté poblado si lo está \texttt{a} o lo está \texttt{b}.
      Es claro observar que el tipo \texttt{Either a b} cumple lo pedido.
      \begin{lstlisting}
  -- Introducción de la disyunción
  disjIntro :: a -> Either a b
  disjIntro x = Left x

  -- Análisis por casos
  either :: (a -> c) -> (b -> c) -> Either a b -> c 
      \end{lstlisting}

    \subsection{Falsum et Verum}
      Nos quedan por definir las proposiciones \textit{Falso} y \textit{Verdadero} y la
      negación de proposiciones. Aquí llegamos al punto en el que el Haskell no sirve exactamente
      para nuestro propósito. En Haskell no se puede definir la función vacía, que sería necesaria
      para la negación. \footnote{Realemente hay una forma bastante lógica de seguir con estas
      definiciones. Pero no es la que luego usaremos en Coq: 
      \url{http://www.haskell.org/haskellwiki/Curry-Howard-Lambek_correspondence}}
      
      \textit{Falso} será un tipo sin constructores, y por tanto, no habitado. De él obtenemos
      la negación y \texttt{Verdadero} mediante implicaciones:
      \begin{lstlisting}
  -- Falso
  data Falsum

  -- La función vacía permite llegar a cualquier tipo
  -- pero la sintaxis de Haskell no permite definirla
  exFalsoQuodlibet :: Falsum -> a

  -- Negar algo es afirmar que implica falsedad
  type Not a = a -> Falsum

  -- Y el tipo Verdadero es, trivialmente
  type Verum = Falsum -> Falsum
      \end{lstlisting}
      Nótese que aquí estamos usando las reglas lógicas:
      \begin{align*}
        \forall A&:& (A \Rightarrow False) &=& (\neg A \vee False) &=& \neg A \\
        \forall A&:& False \Rightarrow False &=& (\neg False \vee False) &=& True
      \end{align*}
      
    \subsection{El isomorfismo en Haskell}
      Los tipos ya definidos y habitados en Haskell (\texttt{Int, Bool, String}) no serían más que axiomas definidos
      previamente.
    
      %% Isomorfismo completo en Haskell
      %% Tipos Axiomáticos
      
    
  \section{Deducción natural}
    En 1935, Gerhard Genzen publicó dos nuevas formulaciones de la lógica, siendo
    una de ellas la \textbf{deducción natural}. Junto a ellas estableció un método
    de simplificación de demostraciones, que servía para asegurarse que la demostración
    no daba vueltas innecesarias.
    
    Para asegurarse de esto, estableció que en la demostración normalizada de una
    fórmula sólo podían aparecer sus subfórmulas. Por ejemplo, para demostrar $A \wedge B$,
    sólo podían usarse $A$, $B$ y sus subexpresiones, nunca algo como $A \vee B$.
    Este método asegura además la consistencia. No existe forma de demostrar $\bot$,
    la proposición contradicción (False).
  
    \subsection{Lógica intuicionista}
      La lógica intuicionista se diferencia de la lógica clásica en que usa una noción
      constructivista de verdad. En particular, el enunciado $A \vee B$ sólo puede ser
      demostrado si se construye alguna prueba de $A$ o de $B$. En consecuencia, la
      ley del tercio excluso $A \vee \neg A$, no puede demostrarse. Aun así,
      puede introducirse como axioma.
      
      La lógica intuicionista es perfecta para ilustrar el isomorfismo, porque
      se corresponde con el cálculo lambda tipado.\\
      
      \begin{tabular}{p{6cm}p{6cm}}
	\begin{prooftree}
	  \AxiomC{A}
	  \AxiomC{B}
	  \RightLabel{($\&$-I)}
	  \BinaryInfC{A \& B}
	\end{prooftree} & 
	\begin{prooftree}
	  \AxiomC{A::M}
	  \AxiomC{B::N}
	  \RightLabel{($\times$-I)}
	  \BinaryInfC{A$\times$B::(M,N)}
	\end{prooftree} \\
	
	\begin{prooftree}
	  \AxiomC{A \& B}
	  \RightLabel{($\&$-$E_1$)}
	  \UnaryInfC{A}
	\end{prooftree} & 
	\begin{prooftree}
	  \AxiomC{A$\times$B::(M,N)}
	  \RightLabel{($\pi_1$)}
	  \UnaryInfC{A::M}
	\end{prooftree} \\
	
	\begin{prooftree}
	  \AxiomC{A \& B}
	  \RightLabel{($\&$-$E_2$)}
	  \UnaryInfC{B}
	\end{prooftree} & 
	\begin{prooftree}
	  \AxiomC{A$\times$B::(M,N)}
	  \RightLabel{($\pi_2$)}
	  \UnaryInfC{B::N}
	\end{prooftree} \\
	
	\begin{prooftree}
	  \AxiomC{A $\rightarrow$ B}
	  \AxiomC{A}
	  \RightLabel{($\rightarrow$-$E$)}
	  \BinaryInfC{B}
	\end{prooftree} & 
	\begin{prooftree}
	  \AxiomC{$\lambda$A.B::M$\rightarrow$N}
	  \AxiomC{A::M}
	  \RightLabel{($\lambda$-$E$)}
	  \BinaryInfC{B::N}
	\end{prooftree} \\
	
	\begin{prooftree}
	  \alwaysNoLine
	  \AxiomC{$[$A$]^x$}
	  \UnaryInfC{$\vdots$}
	  \UnaryInfC{B}
	  \alwaysSingleLine
	  \RightLabel{($\rightarrow$-$I^x$)}
	  \UnaryInfC{A$\rightarrow$B}
	\end{prooftree} & 
	\begin{prooftree}
	  \alwaysNoLine
	  \AxiomC{$[$A::M$]^x$}
	  \UnaryInfC{$\vdots$}
	  \UnaryInfC{B::N}
	  \alwaysSingleLine
	  \RightLabel{($\lambda$-$I^x$)}
	  \UnaryInfC{$\lambda A.B$ :: $M \rightarrow N$}
	\end{prooftree}
      \end{tabular}
  
      El isomorfismo va más allá de las reglas formales. La evaluación de
      programas en el cálculo lambda se corresponde directamente con la simplificación
      y normalización de pruebas que usa el cálculo natural.
  
  \section{Introducción a Coq}
    \subsection{Instalación}
      Los paquetes a instalar para usar el asistente de demostraciones Coq y
      el IDE oficial que permite escribir interactivamente las demostraciones son:
      \begin{lstlisting}
      sudo apt-get install coq coqide coq-theories
      \end{lstlisting}
      Una alternativa más potente y recomendada a CoqIde es ProofGeneral, que funciona 
      como plugin para Emacs.
      
      En ambos puede avanzarse línea a línea sobre las definiciones y demostraciones
      (cada línea en Coq termina en un punto), usando \texttt{C-c C-n} en Emacs y usando
      el comando de avance en CoqIde.
      
%     \subsection{Definición de un tipo}
%       En Coq, los tipos habituales no son nativos (\texttt{int, string, bool}).
%       En lugar de ello, el lenguaje ofrece la capacidad de definir todos estos tipos
%       de datos al usuario y los incluye en librerías.
%       
%       Los tipos se definen por enumeración de sus constructores, que pueden ser
%       parametrizados o no.
      
      %\lstinputlisting[language=Coq]{tipos.v}
  
  %\section{Lógica en Coq}
  %  \subsection{Lógica constructivista}
  
  \vfill
  \begin{thebibliography}{9}

  \bibitem{softwareFund}
    Benjamin Pierce et al.,
    \emph{Software Foundations}.
    University of Pennsylvania,
    2014 (Version 3.1).

  \bibitem{walder}
    Philip Walder,
    \emph{Propositions as Types}.
    University of Edimburg,
    
    
  \bibitem{lectures}
    Morten Heine B. Sørensen, Paweł Urzyczyn,
    \emph{Lectures on the Curry-Howard isomorphism}.
    
    
  \end{thebibliography}

\end{document}
