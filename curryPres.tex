%%%%%
% Curry-Howard
% Author: Mario Román
% License:
% CC BY-NC-SA 3.0 (http://creativecommons.org/licenses/by-nc-sa/3.0/)
%%%%%


%%%%%%%%%%%%%%%%%%%%%%%%%%%%%%%%%%%%%%%%%
% Beamer Presentation
% LaTeX Template
% Version 1.0 (10/11/12)
%
% This template has been downloaded from:
% http://www.LaTeXTemplates.com
%
% License:
% CC BY-NC-SA 3.0 (http://creativecommons.org/licenses/by-nc-sa/3.0/)
%
%%%%%%%%%%%%%%%%%%%%%%%%%%%%%%%%%%%%%%%%%


\documentclass{beamer}

%----------------------------------------------------------------------------------------
% 	Aditional Packages.
%----------------------------------------------------------------------------------------
\usepackage[utf8]{inputenc}
\usepackage[spanish]{babel}
\selectlanguage{spanish}

\usepackage{bussproofs}

\usepackage{listings}
\usepackage{listingsutf8}
% \usepackage[usenames,dvipsnames]{xcolor}
\definecolor{dkgreen}{rgb}{0,0.35,0}
\definecolor{dkviolet}{rgb}{0.3,0,0.5}
\definecolor{dkred}{rgb}{0.5,0,0}
\definecolor{dkblue}{rgb}{0, 0, 0.9}
\definecolor{ltblue}{rgb}{0.38, 0.55, 0.8}
\usepackage{lstcoq}
\newenvironment{coq_example}{\begin{coq}}{\end{coq}}
\lstset{
     literate=%
         {á}{{\'a}}1
         {í}{{\'i}}1
         {é}{{\'e}}1
         {ý}{{\'y}}1
         {ú}{{\'u}}1
         {ó}{{\'o}}1
         {ě}{{\v{e}}}1
         {š}{{\v{s}}}1
         {č}{{\v{c}}}1
         {ř}{{\v{r}}}1
         {ž}{{\v{z}}}1
         {ď}{{\v{d}}}1
         {ť}{{\v{t}}}1
         {ň}{{\v{n}}}1                
         {ů}{{\r{u}}}1
         {Á}{{\'A}}1
         {Í}{{\'I}}1
         {É}{{\'E}}1
         {Ý}{{\'Y}}1
         {Ú}{{\'U}}1
         {Ó}{{\'O}}1
         {Ě}{{\v{E}}}1
         {Š}{{\v{S}}}1
         {Č}{{\v{C}}}1
         {Ř}{{\v{R}}}1
         {Ž}{{\v{Z}}}1
         {Ď}{{\v{D}}}1
         {Ť}{{\v{T}}}1
         {Ň}{{\v{N}}}1                
         {Ů}{{\r{U}}}1    
}

%----------------------------------------------------------------------------------------
%	PACKAGES AND THEMES
%----------------------------------------------------------------------------------------



\mode<presentation> {

% The Beamer class comes with a number of default slide themes
% which change the colors and layouts of slides. Below this is a list
% of all the themes, uncomment each in turn to see what they look like.

%\usetheme{default}
%\usetheme{AnnArbor}
%\usetheme{Antibes}
%\usetheme{Bergen}
%\usetheme{Berkeley}
%\usetheme{Berlin}
%\usetheme{Boadilla}
%\usetheme{CambridgeUS}
%\usetheme{Copenhagen}
%\usetheme{Darmstadt}
%\usetheme{Dresden}
%\usetheme{Frankfurt}
%\usetheme{Goettingen}
%\usetheme{Hannover}
%\usetheme{Ilmenau}
%\usetheme{JuanLesPins}
%\usetheme{Luebeck}
\usetheme{Madrid}
%\usetheme{Malmoe}
%\usetheme{Marburg}
%\usetheme{Montpellier}
%\usetheme{PaloAlto}
%\usetheme{Pittsburgh}
%\usetheme{Rochester}
%\usetheme{Singapore}
%\usetheme{Szeged}
%\usetheme{Warsaw}

% As well as themes, the Beamer class has a number of color themes
% for any slide theme. Uncomment each of these in turn to see how it
% changes the colors of your current slide theme.

%\usecolortheme{albatross}
%\usecolortheme{beaver}
%\usecolortheme{beetle}
%\usecolortheme{crane}
%\usecolortheme{dolphin}
%\usecolortheme{dove}
%\usecolortheme{fly}
%\usecolortheme{lily}
%\usecolortheme{orchid}
\usecolortheme{rose}
%\usecolortheme{seagull}
%\usecolortheme{seahorse}
%\usecolortheme{whale}
%\usecolortheme{wolverine}

%\setbeamertemplate{footline} % To remove the footer line in all slides uncomment this line
%\setbeamertemplate{footline}[page number] % To replace the footer line in all slides with a simple slide count uncomment this line

%\setbeamertemplate{navigation symbols}{} % To remove the navigation symbols from the bottom of all slides uncomment this line
}

\usepackage{graphicx} % Allows including images
\usepackage{booktabs} % Allows the use of \toprule, \midrule and \bottomrule in tables

%----------------------------------------------------------------------------------------
%	TITLE PAGE
%----------------------------------------------------------------------------------------

\title[Isomorfismo Curry-Howard]{El Isomorfismo de Curry-Howard} % The short title appears at the bottom of every slide, the full title is only on the title page

\author{Mario Román} % Your name
\institute[UGR] % Your institution as it will appear on the bottom of every slide, may be shorthand to save space
{
\textit{mromang08+github@gmail.com} \\ \medskip 
Universidad de Granada \\ % Your institution for the title page
\medskip % Your email address
\includegraphics[scale=0.5]{by-sa.png}
}

\date{\today} % Date, can be changed to a custom date

\begin{document}

\begin{frame}
\titlepage % Print the title page as the first slide
\end{frame}

\begin{frame}
\frametitle{Overview} % Table of contents slide, comment this block out to remove it
\tableofcontents % Throughout your presentation, if you choose to use \section{} and \subsection{} commands, these will automatically be printed on this slide as an overview of your presentation
\end{frame}

%----------------------------------------------------------------------------------------
%	PRESENTATION SLIDES
%----------------------------------------------------------------------------------------


\section{El isomorfismo en la deducción natural.}

  \subsection{La deducción natural}
  
    \begin{frame}
    \frametitle{La lógica intuicionista}
      La lógica intuicionista se diferencia de la lógica clásica en que usa una \textbf{noción
      constructivista de verdad}. En particular, el enunciado $A \vee B$ sólo puede ser
      demostrado si se construye alguna prueba de $A$ o de $B$. En consecuencia, la
      ley del tercio excluso $A \vee \neg A$, no puede demostrarse. \\~\\
      
      Aun así, puede introducirse como axioma posteriormente:
      \begin{block}{Axioma del tercio excluso}
	Puede hallarse siempre evidencia de $A$ o de $\neg A$. En particular, para cualquier
	proposición $A$ puede hallarse evidencia de:
	\begin{equation*}
	 A \vee \neg A
	\end{equation*}
	\smallskip
      \end{block}
      
    \end{frame}

  %------------------------------------------------

    \begin{frame}
      \frametitle{La deducción natural}
      En 1935, Gerhard Genzen publicó dos nuevas formulaciones de la lógica intuicionista, siendo
      una de ellas la \textbf{deducción natural}. Junto a ellas estableció un método
      de simplificación de demostraciones, que servía para asegurarse que la demostración
      no daba vueltas innecesarias. \\~\\
      
      Para asegurarse de esto, estableció que en la demostración normalizada de una
      fórmula sólo podían aparecer sus subfórmulas. Por ejemplo, para demostrar $A \wedge B$,
      sólo podían usarse $A$, $B$ y sus subexpresiones, nunca algo como $A \vee B$.
      Este método asegura además la consistencia. No existe forma de demostrar $\bot$,
      la proposición contradicción (False).
    \end{frame}

  %------------------------------------------------

    \begin{frame}
      \frametitle{Reglas de la deducción natural}
      La deducción natural posee las siguientes reglas:
      
      \texttt{
      \begin{tabular}{p{5cm}|p{5cm}}
	\begin{prooftree}
	  \AxiomC{A}
	  \AxiomC{B}
	  \RightLabel{($\&-I$)}
	  \BinaryInfC{A \& B}
	\end{prooftree} \\
	\hline
	\begin{prooftree}
	  \AxiomC{A \& B}
	  \RightLabel{($\&$-$E_1$)}
	  \UnaryInfC{A}
	\end{prooftree} &
	\begin{prooftree}
	  \AxiomC{A \& B}
	  \RightLabel{($\&$-$E_2$)}
	  \UnaryInfC{B}
	\end{prooftree} \\
	\hline
	\begin{prooftree}
	  \AxiomC{A $\rightarrow$ B}
	  \AxiomC{A}
	  \RightLabel{($\rightarrow$-$E$)}
	  \BinaryInfC{B}
	\end{prooftree} &
	\begin{prooftree}
	  \alwaysNoLine
	  \AxiomC{$[$A$]^x$}
	  \UnaryInfC{$\vdots$}
	  \UnaryInfC{B}
	  \alwaysSingleLine
	  \RightLabel{($\rightarrow$-$I^x$)}
	  \UnaryInfC{A$\rightarrow$B}
	\end{prooftree}  
      \end{tabular}
      }
    \end{frame}    
        
    
  %------------------------------------------------

  \subsection{El cálculo lambda tipado}
    \begin{frame}
      \frametitle{El cálculo lambda tipado}
      En 1940, Alonzo Church publicó una nueva formulación del cálculo lambda, siendo esta
      el cálculo lambda tipado. Junto a ella podía usarse la simplificación de programas
      con estrategias de reducción. \\~\\
      
      Para asegurarse de que el cálculo lambda no hacía aparecer ciertas paradojas,
      introdujo los tipos y limitó los constructores de tipos a uno solo: el operador
      $\rightarrow$, que construye el tipo función entre dos tipos. Por esto, puede denotarse
      por cálculo $\lambda^{\rightarrow}$.
    \end{frame}

  %------------------------------------------------

  
    \begin{frame}
      \frametitle{Reglas del cálculo lambda tipado}
      El cálculo lambda tipado posee las siguientes reglas:
      
      \ttfamily
      \begin{tabular}{p{5cm}|p{5cm}}
	\begin{prooftree}
	  \AxiomC{A::M}
	  \AxiomC{B::N}
	  \RightLabel{($\times-I$)}
	  \BinaryInfC{A$\times$B::(M,N)}
	\end{prooftree} \\
	\hline
	
	\begin{prooftree}
	  \AxiomC{A$\times$B::(M,N)}
	  \RightLabel{($\pi_1$)}
	  \UnaryInfC{A::M}
	\end{prooftree} &
	
	\begin{prooftree}
	  \AxiomC{A$\times$B::(M,N)}
	  \RightLabel{($\pi_2$)}
	  \UnaryInfC{B::N}
	\end{prooftree} \\
	\hline
	
	\begin{prooftree}
	  \AxiomC{$\lambda$A.B::M$\rightarrow$N}
	  \AxiomC{A::M}
	  \RightLabel{($\lambda$-$E$)}
	  \BinaryInfC{B::N}
	\end{prooftree} &
	
	\begin{prooftree}
	  \alwaysNoLine
	  \AxiomC{$[$A::M$]^x$}
	  \UnaryInfC{$\vdots$}
	  \UnaryInfC{B::N}
	  \alwaysSingleLine
	  \RightLabel{($\lambda$-$I^x$)}
	  \UnaryInfC{$\lambda A.B$ :: $M \rightarrow N$}
	\end{prooftree}
      \end{tabular}
    \end{frame}
    
  %------------------------------------------------

  \subsection{Conclusión}
    \begin{frame}
      \frametitle{Las proposiciones son tipos}
      De todo esto sacamos una conclusión: \textbf{las proposiciones lógicas son tipos}. 
      A cada proposición lógica le corresponde un tipo, unívocamente. 
      Y una proposición será verdadera si y sólo si su tipo correspondiente está
      habitado. Es decir, existe alguna instancia de ese tipo. \\~\\
      
      \begin{itemize}
       \item Los tipos conocidos y habitados son los \textbf{axiomas} (\texttt{int}, \texttt{char}, \texttt{string}).
       \item Una función de tipos hace que uno implique otro (\texttt{f :: a -> b}, si \texttt{a} está poblado por \texttt{x}, \texttt{f(x)} puebla b).
	Es la \textbf{implicación} lógica.
       \item El producto está poblado ssi ambos lo están (hay un objeto \texttt{(a,b)} por cada pareja \texttt{a} y \texttt{b}).
	Es la \textbf{conjunción} lógica.
      \end{itemize}
      
    \end{frame}
    
  %------------------------------------------------
  
%     \begin{frame}
%     \frametitle{Bullet Points}
%     \begin{itemize}
%     \item Lorem ipsum dolor sit amet, consectetur adipiscing elit
%     \item Aliquam blandit faucibus nisi, sit amet dapibus enim tempus eu
%     \item Nulla commodo, erat quis gravida posuere, elit lacus lobortis est, quis porttitor odio mauris at libero
%     \item Nam cursus est eget velit posuere pellentesque
%     \item Vestibulum faucibus velit a augue condimentum quis convallis nulla gravida
%     \end{itemize}
%     \end{frame}

  %------------------------------------------------

\section{El isomorfismo en Haskell}
\begin{frame}[fragile] % Need to use the fragile option when verbatim is used in the slide
  \frametitle{El isomorfismo en Haskell}
  El sistema de tipos de Haskell permite denotar por \texttt{a} un
  tipo arbitrario. Denota con \texttt{->} el tipo de las funciones entre
  dos elementos. Sólo algunos tipos están habitados.
  \begin{block}{Tipos permitidos}
  \begin{lstlisting}[language=Haskell]
  a -> a                -- identity x = x
  a -> (b -> a)         -- const a = (\b->a)
  (a,b) -> a            -- proy (a,b) = a
  (a, (a -> b)) -> b    -- apply f x = f(x) \end{lstlisting}
  \end{block}
  \begin{block}{Tipos no permitidos}
    \begin{lstlisting}
  a -> (a -> b)
  a -> b
  b -> (b,c)
    \end{lstlisting}
  \end{block}
\end{frame}

   %------------------------------------------------
  
\section{El isomorfismo en Coq}
\begin{frame}[fragile]
  \frametitle{El isomorfismo en Coq}
    El asistente de demostraciones Coq consiste en un lenguaje de
    programación y un intérprete del lenguaje destinados a demostrar
    proposiciones lógicas. Para esto, usa el isomorfismo en sentido inverso,
    un objeto de un tipo determinado es una demostración de una proposición
    determinada. \\~\\
    
    \begin{block}{Demostración en Coq}
     \begin{lstlisting}[language=Coq]
(* Theorem / Type declaration*)
Theorem eight_is_beautiful : beautiful 8.
(* Proof / Construction *)
Proof.
  apply b_sum with (n:=3) (m:=5).
  apply b_3.
  apply b_5.
Qed.
     \end{lstlisting}

    \end{block}

 
\end{frame}

% 
%   %------------------------------------------------
% 
%   \begin{frame}
%   \frametitle{Multiple Columns}
%   \begin{columns}[c] % The "c" option specifies centered vertical alignment while the "t" option is used for top vertical alignment
% 
%   \column{.45\textwidth} % Left column and width
%   \textbf{Heading}
%   \begin{enumerate}
%   \item Statement
%   \item Explanation
%   \item Example
%   \end{enumerate}
% 
%   \column{.5\textwidth} % Right column and width
%   Lorem ipsum dolor sit amet, consectetur adipiscing elit. Integer lectus nisl, ultricies in feugiat rutrum, porttitor sit amet augue. Aliquam ut tortor mauris. Sed volutpat ante purus, quis accumsan dolor.
% 
%   \end{columns}
%   \end{frame}
% 
% \begin{frame}
% \frametitle{Table}
% \begin{table}
% \begin{tabular}{l l l}
% \toprule
% \textbf{Treatments} & \textbf{Response 1} & \textbf{Response 2}\\
% \midrule
% Treatment 1 & 0.0003262 & 0.562 \\
% Treatment 2 & 0.0015681 & 0.910 \\
% Treatment 3 & 0.0009271 & 0.296 \\
% \bottomrule
% \end{tabular}
% \caption{Table caption}
% \end{table}
% \end{frame}
% 
% %------------------------------------------------
% 
% \begin{frame}
% \frametitle{Theorem}
% \begin{theorem}[Mass--energy equivalence]
% $E = mc^2$
% \end{theorem}
% \end{frame}
% 
% %------------------------------------------------
% 

% 
% %------------------------------------------------
% 
% \begin{frame}
% \frametitle{Figure}
% Uncomment the code on this slide to include your own image from the same directory as the template .TeX file.
% %\begin{figure}
% %\includegraphics[width=0.8\linewidth]{test}
% %\end{figure}
% \end{frame}

%------------------------------------------------

\begin{frame}
\frametitle{References}
\footnotesize{
\begin{thebibliography}{99} % Beamer does not support BibTeX so references must be inserted manually as below

\bibitem[softwareFund]{p2} Benjamin Pierce et al. (2014)
  \newblock Software Foundations.
  \newblock \emph{University of Pennsylvania}

\bibitem[walder]{p3} Philip Walder
  \newblock Propositions as Types
  \newblock \emph{University of Edimburg}

\bibitem[lectures]{p4} Morten Heine B. Sørensen, Paweł Urzyczyn,
  \newblock Lectures on the Curry-Howard isomorphism
  \newblock \emph{Universities of Warsaw and Copenhagen}

\end{thebibliography}
}
\end{frame}

%----------------------------------------------------------------------------------------

\end{document} 